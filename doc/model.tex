
% -------------------------------------------
% Items to substitute into the ivoatex document template.
%
%\ivoagroup{Data Model Working Group}

%\title{Mango}


%\author{Laurent Michel}
    
%\author{François Bonnarel}
    
%\author{Gilles Landais}
    
%\author{Mireille Louys}
    
%\author{Marco Molinaro}
    
%\author{Jesue Salgado}
    
%\previousversion{0.0}
      
% -------------------------------------------

\pagebreak
\section{Model: mango }
  
  % INSERT FIGURE HERE
  %\begin{figure}[h]
  %\begin{center}
  %  \includegraphics[width=\textwidth]{????.png}
  %  \caption{???}\label{fig:????}
  %\end{center}
  %\end{figure}

  Data model based oon components and data association for source data

  \subsection{AssociatedData (Abstract)}
  \label{sect:AssociatedData}
    Abstract reference to a particular dataset associated to the Source. This class is used to specify the type of the dataset as well as its role.

    \subsubsection{AssociatedData.semantic}
      \textbf{vodml-id: AssociatedData.semantic} \newline
      \textbf{type: \hyperref[sect:VocabularyTerm]{mango:VocabularyTerm}} \newline
      \textbf{multiplicity: 1} \newline 
      Reference to a semantic concept giving the nature of the associated data. As long as the vocabulary is not set, the possible values of this attribute are given by the LinkSemantic enumeration.

    \subsubsection{AssociatedData.dataType}
      \textbf{vodml-id: AssociatedData.dataType} \newline
      \textbf{type: \hyperref[sect:ivoa]{ivoa:string}} \newline
      \textbf{multiplicity: 1} \newline 
      Type of the associated data (not defined yet)

    \subsubsection{AssociatedData.description}
      \textbf{vodml-id: AssociatedData.description} \newline
      \textbf{type: \hyperref[sect:ivoa]{ivoa:string}} \newline
      \textbf{multiplicity: 1} \newline 
      Free text description of the associated data

  \subsection{AssociatedMangoInstance}
  \label{sect:AssociatedMangoInstance}
    Reference to another MANGO instance that is part of the associated data.

    \subsubsection{AssociatedMangoInstance.mangoInstance}
      \textbf{vodml-id: AssociatedMangoInstance.mangoInstance} \newline
      \textbf{type: \hyperref[sect:Source]{mango:Source}} \newline
      \textbf{multiplicity: 1} \newline 
      Composition link pointing on one MANGO instance associated with the source.

  \subsection{ModelInstance}
  \label{sect:ModelInstance}
    Placeholder for the mapping of the model instance

  \subsection{Parameter}
  \label{sect:Parameter}
    Reference to a particular measure of the Source. This class is used to specify the type of the measure as well as its role.

    \noindent \textbf{constraint} \newline
    \indent    \textbf{detail: Parameter.One association at the time
 }\newline


    \subsubsection{Parameter.semantic}
      \textbf{vodml-id: Parameter.semantic} \newline
      \textbf{type: \hyperref[sect:VocabularyTerm]{mango:VocabularyTerm}} \newline
      \textbf{multiplicity: 1} \newline 
      Reference to a semantic concept giving the nature of the parameter As long as the vocabulary is not set, the possible values of this attribute are given by the ParamSemantic enumeration.

    \subsubsection{Parameter.ucd}
      \textbf{vodml-id: Parameter.ucd} \newline
      \textbf{type: \hyperref[sect:ivoa]{ivoa:string}} \newline
      \textbf{multiplicity: 1} \newline 
      UCD1+ giving the type of the physical measure

    \subsubsection{Parameter.description}
      \textbf{vodml-id: Parameter.description} \newline
      \textbf{type: \hyperref[sect:ivoa]{ivoa:string}} \newline
      \textbf{multiplicity: 1} \newline 
      Free text description of the measure

    \subsubsection{Parameter.measure}
      \textbf{vodml-id: Parameter.measure} \newline
      \textbf{type: meas:Measure} \newline
      \textbf{multiplicity: 1} \newline 
      Composition link pointing to the meas:Measure instance

    \subsubsection{Parameter.associatedParameters}
      \textbf{vodml-id: Parameter.associatedParameters} \newline
      \textbf{type: \hyperref[sect:Parameter]{mango:Parameter}} \newline
      \textbf{multiplicity: 0..*} \newline 
      This association allows to gather different parameters without explicit semantic. It can be used to attach a quality flags to a parameter. It can be used to put together the parameters whose errors are coupled together. Another case could be to associate a planet description with its orbital parameters. Using associatedParameters allow to limit the number of complex classes to be implemented in the model.

  \subsection{Source}
  \label{sect:Source}
    Root class of the model. MANGO instance are meant of be Source instances. A source has an identifier and two docks: one for the parameters and one for the associated data.

    \subsubsection{Source.identifier}
      \textbf{vodml-id: Source.identifier} \newline
      \textbf{type: \hyperref[sect:ivoa]{ivoa:string}} \newline
      \textbf{multiplicity: 1} \newline 
      Unique identifier for a Source. The uniqueness of that identifier is not managed by the model. The format is free.

    \subsubsection{Source.associatedDataDock}
      \textbf{vodml-id: Source.associatedDataDock} \newline
      \textbf{type: \hyperref[sect:AssociatedData]{mango:AssociatedData}} \newline
      \textbf{multiplicity: 0..*} \newline 
      Composition link pointing on all data associated with the source.

    \subsubsection{Source.parameterDock}
      \textbf{vodml-id: Source.parameterDock} \newline
      \textbf{type: \hyperref[sect:Parameter]{mango:Parameter}} \newline
      \textbf{multiplicity: 0..*} \newline 
      Composition link pointing on all parameters attached to the source.

  \subsection{VOModelInstance}
  \label{sect:VOModelInstance}
    Reference to a VO model instance that is part of the associated data.

    \subsubsection{VOModelInstance.ivoid}
      \textbf{vodml-id: VOModelInstance.ivoid} \newline
      \textbf{type: \hyperref[sect:ivoa]{ivoa:string}} \newline
      \textbf{multiplicity: 1} \newline 
      VO-DML id of the referenced model

    \subsubsection{VOModelInstance.modelUrl}
      \textbf{vodml-id: VOModelInstance.modelUrl} \newline
      \textbf{type: \hyperref[sect:ivoa]{ivoa:anyURI}} \newline
      \textbf{multiplicity: 1} \newline 
      URL on the VO-DML model

    \subsubsection{VOModelInstance.modelName}
      \textbf{vodml-id: VOModelInstance.modelName} \newline
      \textbf{type: \hyperref[sect:ivoa]{ivoa:string}} \newline
      \textbf{multiplicity: 1} \newline 
      Name of the referenced model

    \subsubsection{VOModelInstance.modelDoc}
      \textbf{vodml-id: VOModelInstance.modelDoc} \newline
      \textbf{type: \hyperref[sect:ivoa]{ivoa:anyURI}} \newline
      \textbf{multiplicity: 1} \newline 
      Documentation URL of the model

    \subsubsection{VOModelInstance.modelInstance}
      \textbf{vodml-id: VOModelInstance.modelInstance} \newline
      \textbf{type: \hyperref[sect:ModelInstance]{mango:ModelInstance}} \newline
      \textbf{multiplicity: 1} \newline 
      Composition link pointing on one VO instance instance associated with the source.

  \subsection{VOService}
  \label{sect:VOService}
    Class for associated data referenced by a fixed URL that is a VO service.

    \subsubsection{VOService.ivoid}
      \textbf{vodml-id: VOService.ivoid} \newline
      \textbf{type: \hyperref[sect:ivoa]{ivoa:string}} \newline
      \textbf{multiplicity: 1} \newline 
      IVOA id of the service (for example in the registry)

  \subsection{VocabularyTerm}
  \label{sect:VocabularyTerm}
    Datatype for vocabulary word. Provides a pointer to the word description and a label.

    \subsubsection{VocabularyTerm.uri}
      \textbf{vodml-id: VocabularyTerm.uri} \newline
      \textbf{type: \hyperref[sect:ivoa]{ivoa:string}} \newline
      \textbf{multiplicity: 1} \newline 
      URI extracted from the DRF document and referring to the word

    \subsubsection{VocabularyTerm.label}
      \textbf{vodml-id: VocabularyTerm.label} \newline
      \textbf{type: \hyperref[sect:ivoa]{ivoa:string}} \newline
      \textbf{multiplicity: 1} \newline 
      RDF label. Matched the URL fragment for IVOA vocabularies

  \subsection{WebEndpoint}
  \label{sect:WebEndpoint}
    Class for associated data referenced by an URL

    \subsubsection{WebEndpoint.ContentType}
      \textbf{vodml-id: WebEndpoint.ContentType} \newline
      \textbf{type: \hyperref[sect:ivoa]{ivoa:string}} \newline
      \textbf{multiplicity: 1} \newline 
      Mime type of the URL

    \subsubsection{WebEndpoint.url}
      \textbf{vodml-id: WebEndpoint.url} \newline
      \textbf{type: \hyperref[sect:ivoa]{ivoa:anyURI}} \newline
      \textbf{multiplicity: 1} \newline 
      Web endpoint

  \subsection{LinkSemantic}
  \label{sect:LinkSemantic}

  Literal enumeration of the possible values for the associated data semantic. This stands for an example before we have defined a vocabulary.

  \noindent \underline{Enumeration Literals}
  \vspace{-\parsep}
  \small
  \begin{itemize}
  
    \item[\textbf{VOService}]: \textbf{vodml-id:} LinkSemantic.VOService \newline
          \textbf{description:} Data returned by a VO service
    \item[\textbf{VOInstance}]: \textbf{vodml-id:} LinkSemantic.VOInstance \newline
          \textbf{description:} Data Serialized in a VO model
    \item[\textbf{Preview}]: \textbf{vodml-id:} LinkSemantic.Preview \newline
          \textbf{description:} data preview
    \item[\textbf{DownloadLink}]: \textbf{vodml-id:} LinkSemantic.DownloadLink \newline
          \textbf{description:} Data download link
    \item[\textbf{Detection}]: \textbf{vodml-id:} LinkSemantic.Detection \newline
          \textbf{description:} Particular detection
    \item[\textbf{Compagnon}]: \textbf{vodml-id:} LinkSemantic.Compagnon \newline
          \textbf{description:} Compagnon source
    \item[\textbf{Counterpart}]: \textbf{vodml-id:} LinkSemantic.Counterpart \newline
          \textbf{description:} Counter part source
  \end{itemize}
  \normalsize


  \subsection{ParamSemantic}
  \label{sect:ParamSemantic}

  Literal enumeration of the possible values for the parameter semantic. This stands for an example before we have defined a vocabulary.

  \noindent \underline{Enumeration Literals}
  \vspace{-\parsep}
  \small
  \begin{itemize}
  
    \item[\textbf{Main}]: \textbf{vodml-id:} ParamSemantic.Main \newline
          \textbf{description:} Main measurment
    \item[\textbf{Computed}]: \textbf{vodml-id:} ParamSemantic.Computed \newline
          \textbf{description:} Computed measurement
    \item[\textbf{Native}]: \textbf{vodml-id:} ParamSemantic.Native \newline
          \textbf{description:} Mative measurement
    \item[\textbf{Raw}]: \textbf{vodml-id:} ParamSemantic.Raw \newline
          \textbf{description:} raw measure
    \item[\textbf{Corrected}]: \textbf{vodml-id:} ParamSemantic.Corrected \newline
          \textbf{description:} Corrected measure
  \end{itemize}
  \normalsize


  \subsection{ShapeFrame}
  \label{sect:ShapeFrame}

  Possible options to encode a shape in a string.

  \noindent \underline{Enumeration Literals}
  \vspace{-\parsep}
  \small
  \begin{itemize}
  
    \item[\textbf{STC\_S}]: \textbf{vodml-id:} ShapeFrame.STC\_S \newline
          \textbf{description:} STCs serialisation
    \item[\textbf{MOC}]: \textbf{vodml-id:} ShapeFrame.MOC \newline
          \textbf{description:} MOC serialisation
  \end{itemize}
  \normalsize


\pagebreak
\section{Package: exterrors }

  % INSERT FIGURE HERE
  %\begin{figure}[h]
  %\begin{center}
  %  \includegraphics[width=\textwidth]{????.png}
  %  \caption{???}\label{fig:????}
  %\end{center}
  %\end{figure}

  This package contains all \texttt{meas:Error} class extensions

  \subsection{CorrMatrix2x2}
  \label{sect:exterrors.CorrMatrix2x2}
    Variance matrix with correlation between errors on individual axes.

    \subsubsection{CorrMatrix2x2.correlation}
      \textbf{vodml-id: exterrors.CorrMatrix2x2.correlation} \newline
      \textbf{type: \hyperref[sect:ivoa]{ivoa:real}} \newline
      \textbf{multiplicity: 1} \newline 
      Correlation between the errors on the 2 axes. The covariance is given by $cov_{XY}=corr_{XY}\sigma_{X}\sigma_{Y}$

    \subsubsection{CorrMatrix2x2.diagMatrix}
      \textbf{vodml-id: exterrors.CorrMatrix2x2.diagMatrix} \newline
      \textbf{type: \hyperref[sect:exterrors.DiagElems2x2]{mango:exterrors.DiagElems2x2}} \newline
      \textbf{multiplicity: 1} \newline 
      Diagonal elements of the matrix. The unit is given by \texttt{mango:error.MultiParamError.unit}

  \subsection{Correlated2D1D}
  \label{sect:exterrors.Correlated2D1D}
    Correlation coefficients between the error of a 1D host parameter and a 2D associated parameters.

    \subsubsection{Correlated2D1D.correlation2\_1}
      \textbf{vodml-id: exterrors.Correlated2D1D.correlation2\_1} \newline
      \textbf{type: \hyperref[sect:ivoa]{ivoa:real}} \newline
      \textbf{multiplicity: 1} \newline 
      Correlation between the error on the first axis of the host parameter and the error on the second axis of the associated parameter. The covariance is given by $cov_{XY}=corr_{XY}\sigma_{X}\sigma_{Y}$

  \subsection{CorrelatedError}
  \label{sect:exterrors.CorrelatedError}
    Correlation coefficients between the error of the host parameter and one of its associated parameters. The host parameter is one of the \texttt{mango:Parameter} of the \texttt{mango:ParameterDock} (a position typically) of the Mango object. The associated parameter is one of the \texttt{mango:Parameter.associatedParameters} of that parameter (typically a proper motion) There is no logical link between the correlated error instance and the associated parameter it refers to. The associated parameter is identified by the \texttt{UCD field}. The client is in charge of solving this dependency.

    \subsubsection{CorrelatedError.ucd}
      \textbf{vodml-id: exterrors.CorrelatedError.ucd} \newline
      \textbf{type: \hyperref[sect:ivoa]{ivoa:string}} \newline
      \textbf{multiplicity: 1} \newline 
      UCD of the associated parameter. This UCD must be identical to this of the associated parameter the \texttt{CorrelatedError} refers to.

  \subsection{CorrelatedError1D1D}
  \label{sect:exterrors.CorrelatedError1D1D}
    Correlation coefficients between the error of one 1D host parameter and a one 1D associated parameters.

    \subsubsection{CorrelatedError1D1D.correlation1\_1}
      \textbf{vodml-id: exterrors.CorrelatedError1D1D.correlation1\_1} \newline
      \textbf{type: \hyperref[sect:ivoa]{ivoa:real}} \newline
      \textbf{multiplicity: 1} \newline 
      Correlation between the error on the first axis of the host parameter and the error on the first axis of the associated parameter. The covariance is given by $cov_{XY}=corr_{XY}\sigma_{X}\sigma_{Y}$

  \subsection{CorrelatedError1D2D}
  \label{sect:exterrors.CorrelatedError1D2D}
    Correlation coefficients between the error of one 2D host parameter and one 1D associated parameters.

    \subsubsection{CorrelatedError1D2D.correlation1\_1}
      \textbf{vodml-id: exterrors.CorrelatedError1D2D.correlation1\_1} \newline
      \textbf{type: \hyperref[sect:ivoa]{ivoa:real}} \newline
      \textbf{multiplicity: 1} \newline 
      Correlation between the error on the first axis of the host parameter and the error on the associated parameter. The covariance is given by $cov_{XY}=corr_{XY}\sigma_{X}\sigma_{Y}$

    \subsubsection{CorrelatedError1D2D.correlation1\_2}
      \textbf{vodml-id: exterrors.CorrelatedError1D2D.correlation1\_2} \newline
      \textbf{type: \hyperref[sect:ivoa]{ivoa:real}} \newline
      \textbf{multiplicity: 1} \newline 
      Correlation between the error on the second axis of the host parameter and the error on the associated parameter. The covariance is given by $cov_{XY}=corr_{XY}\sigma_{X}\sigma_{Y}$

  \subsection{CorrelatedError2D2D}
  \label{sect:exterrors.CorrelatedError2D2D}
    Correlation coefficients between the error of a 2D host parameter and a 2D associated parameters.

    \subsubsection{CorrelatedError2D2D.correlation2\_1}
      \textbf{vodml-id: exterrors.CorrelatedError2D2D.correlation2\_1} \newline
      \textbf{type: \hyperref[sect:ivoa]{ivoa:real}} \newline
      \textbf{multiplicity: 1} \newline 
      Correlation between the error on the first axis of the host parameter and the error on the second axis of the associated parameter. The covariance is given by $cov_{XY}=corr_{XY}\sigma_{X}\sigma_{Y}$

    \subsubsection{CorrelatedError2D2D.correlation2\_2}
      \textbf{vodml-id: exterrors.CorrelatedError2D2D.correlation2\_2} \newline
      \textbf{type: \hyperref[sect:ivoa]{ivoa:real}} \newline
      \textbf{multiplicity: 1} \newline 
      Correlation between the error on the second axis of the host parameter and the error on the second axis of the associated parameter. The covariance is given by $cov_{XY}=corr_{XY}\sigma_{X}\sigma_{Y}$

  \subsection{DiagMatrix2x2}
  \label{sect:exterrors.DiagMatrix2x2}
    Simple diagonal matrix of variances

    \subsubsection{DiagMatrix2x2.diagMatrix}
      \textbf{vodml-id: exterrors.DiagMatrix2x2.diagMatrix} \newline
      \textbf{type: \hyperref[sect:exterrors.DiagElems2x2]{mango:exterrors.DiagElems2x2}} \newline
      \textbf{multiplicity: 1} \newline 
      Diagonal elements of the matrix. The unit is given by \texttt{mango:errors.MultiParamError.unit}

  \subsection{Ellipse}
  \label{sect:exterrors.Ellipse}
    Elliptical error. The regular ellipse orientation is East of the North

    \subsubsection{Ellipse.semiMajorAxis}
      \textbf{vodml-id: exterrors.Ellipse.semiMajorAxis} \newline
      \textbf{type: \hyperref[sect:ivoa]{ivoa:real}} \newline
      \textbf{multiplicity: 1} \newline 
      Semi major axis of the ellipse. The unit is given by \texttt{mango:errors.MultiParamError.unit}

    \subsubsection{Ellipse.semiMinorAxis}
      \textbf{vodml-id: exterrors.Ellipse.semiMinorAxis} \newline
      \textbf{type: \hyperref[sect:ivoa]{ivoa:real}} \newline
      \textbf{multiplicity: 1} \newline 
      Semi minor axis of the ellipse. The unit is given by \texttt{mango:errors.MultiParamError.unit}

    \subsubsection{Ellipse.positionAngle}
      \textbf{vodml-id: exterrors.Ellipse.positionAngle} \newline
      \textbf{type: \hyperref[sect:ivoa]{ivoa:RealQuantity}} \newline
      \textbf{multiplicity: 1} \newline 
      Ellipse orientation, relative to the East of the North.

  \subsection{MultParamErro2D (Abstract)}
  \label{sect:exterrors.MultParamErro2D}
    This class models errors on a one 2 axes parameter with possible correlations with errors on associated parameters. A classical use-case is an error on a position that is coupled with errors on the proper motion and/or the parralax.

    \noindent \textbf{subset} \newline
    \indent   \textbf{role: mango:exterrors.MultiParamError.correlatedErrrors} \newline
    \indent   \textbf{type:  mango:exterrors.CorrelatedError1D2D} \newline


  \subsection{MultiParamError (Abstract)}
  \label{sect:exterrors.MultiParamError}
    This class models errors with possible correlations between different axes and with errors of associated parameters. The standard use-case for such errors is a positional error where e.g. errors on position, proper motion and parallax are correlated to each other.

    \subsubsection{MultiParamError.confidenceLevel}
      \textbf{vodml-id: exterrors.MultiParamError.confidenceLevel} \newline
      \textbf{type: \hyperref[sect:ivoa]{ivoa:integer}} \newline
      \textbf{multiplicity: 1} \newline 
      Error confidence level, expressed in $\sigma$.

    \subsubsection{MultiParamError.unit}
      \textbf{vodml-id: exterrors.MultiParamError.unit} \newline
      \textbf{type: \hyperref[sect:ivoa]{ivoa:Unit}} \newline
      \textbf{multiplicity: 1} \newline 
      Unit of the variance or the error ellipse

    \subsubsection{MultiParamError.correlatedErrors}
      \textbf{vodml-id: exterrors.MultiParamError.correlatedErrors} \newline
      \textbf{type: \hyperref[sect:exterrors.CorrelatedError]{mango:exterrors.CorrelatedError}} \newline
      \textbf{multiplicity: 0..*} \newline 
      Relation linking the host parameter error with all correlated errors.

  \subsection{MultiParamError1D}
  \label{sect:exterrors.MultiParamError1D}
    This class models errors on a one axis parameter with possible correlations with errors on associated parameters. A classical use-case is an error on a radial velocity that is coupled with an error on the proper motion.

    \noindent \textbf{subset} \newline
    \indent   \textbf{role: mango:errors.MultiParamError.correlatedErrors} \newline
    \indent   \textbf{type:  mango:exterrors.CorrelatedError1D1D} \newline


    \subsubsection{MultiParamError1D.sigma}
      \textbf{vodml-id: exterrors.MultiParamError1D.sigma} \newline
      \textbf{type: \hyperref[sect:ivoa]{ivoa:real}} \newline
      \textbf{multiplicity: 1} \newline 
      Variance of the parameter error. The unit is given by \texttt{mango:errors.MultiParamError.unit}

  \subsection{DiagElems2x2}
  \label{sect:exterrors.DiagElems2x2}
  Datatype containing the 2 diagonal elements of a 2x2 matrix. Attributes are named $\sigma$ because this datatype is mostly used in the context of complex errors.

\pagebreak
\section{Package: extmeas }

  % INSERT FIGURE HERE
  %\begin{figure}[h]
  %\begin{center}
  %  \includegraphics[width=\textwidth]{????.png}
  %  \caption{???}\label{fig:????}
  %\end{center}
  %\end{figure}

  This package contains all \texttt{meas} class extensions

  \subsection{Flag}
  \label{sect:extmeas.Flag}
    Measure to be used for status parameters

    \subsubsection{Flag.coord}
      \textbf{vodml-id: extmeas.Flag.coord} \newline
      \textbf{type: \hyperref[sect:extcoords.FlagCoord]{mango:extcoords.FlagCoord}} \newline
      \textbf{multiplicity: 1} \newline 
      Coordinate holding the status value

  \subsection{GenericStringMeasure}
  \label{sect:extmeas.GenericStringMeasure}
    Generic measure which value is a string. This can be typically used for identifiers or classification values.

    \subsubsection{GenericStringMeasure.coord}
      \textbf{vodml-id: extmeas.GenericStringMeasure.coord} \newline
      \textbf{type: \hyperref[sect:extcoords.StringCoordinate]{mango:extcoords.StringCoordinate}} \newline
      \textbf{multiplicity: 1} \newline 
      TODO : Missing description : please, update your UML model asap.

  \subsection{HardnessRatio}
  \label{sect:extmeas.HardnessRatio}
    Hardness ratio of Energy bands (e.g.1 and 2) $HR_{12} = (Flux_2 - Flux_1)/ (Flux_2 + Flux_1)$ The hardness ration value is a real The coordinate frame is the description of the both bands.

    \subsubsection{HardnessRatio.coord}
      \textbf{vodml-id: extmeas.HardnessRatio.coord} \newline
      \textbf{type: \hyperref[sect:extcoords.HardnessRatioCoord]{mango:extcoords.HardnessRatioCoord}} \newline
      \textbf{multiplicity: 1} \newline 
      TODO : Missing description : please, update your UML model asap.

  \subsection{LonLatSkyPosition}
  \label{sect:extmeas.LonLatSkyPosition}
    Measure to used for sky points expressed with a spherical coordinate system

    \subsubsection{LonLatSkyPosition.coord}
      \textbf{vodml-id: extmeas.LonLatSkyPosition.coord} \newline
      \textbf{type: \hyperref[sect:extcoords.LonLatPoint]{mango:extcoords.LonLatPoint}} \newline
      \textbf{multiplicity: 1} \newline 
      Coordinate of spherical sky position

  \subsection{Orbit}
  \label{sect:extmeas.Orbit}
    TODO : Missing description : please, update your UML model asap.

    \subsubsection{Orbit.coord}
      \textbf{vodml-id: extmeas.Orbit.coord} \newline
      \textbf{type: \hyperref[sect:extcoords.OrbitCoord]{mango:extcoords.OrbitCoord}} \newline
      \textbf{multiplicity: 1} \newline 
      TODO : Missing description : please, update your UML model asap.

  \subsection{Photometry}
  \label{sect:extmeas.Photometry}
    Measure for source luminosity, can be used for magnitude or fluxes either.

    \subsubsection{Photometry.coord}
      \textbf{vodml-id: extmeas.Photometry.coord} \newline
      \textbf{type: \hyperref[sect:extcoords.PhotometryCoord]{mango:extcoords.PhotometryCoord}} \newline
      \textbf{multiplicity: 1} \newline 
      Coordinate of the photometric measure

  \subsection{Redshift}
  \label{sect:extmeas.Redshift}
    TODO : Missing description : please, update your UML model asap.

    \subsubsection{Redshift.coord}
      \textbf{vodml-id: extmeas.Redshift.coord} \newline
      \textbf{type: \hyperref[sect:extcoords.RedshiftCoord]{mango:extcoords.RedshiftCoord}} \newline
      \textbf{multiplicity: 1} \newline 
      TODO : Missing description : please, update your UML model asap.

  \subsection{Shape}
  \label{sect:extmeas.Shape}
    Measure giving the shape of a source. The shape is a string with a specific frame (STS\_S or MOC)

    \subsubsection{Shape.coord}
      \textbf{vodml-id: extmeas.Shape.coord} \newline
      \textbf{type: \hyperref[sect:extcoords.ShapeCoord]{mango:extcoords.ShapeCoord}} \newline
      \textbf{multiplicity: 1} \newline 
      String serialization of the source shape

  \subsection{VocabGenericMeasure}
  \label{sect:extmeas.VocabGenericMeasure}
    Generic measure which value is a vocabulary entry. This can be typically used for object types.

    \subsubsection{VocabGenericMeasure.coord}
      \textbf{vodml-id: extmeas.VocabGenericMeasure.coord} \newline
      \textbf{type: \hyperref[sect:extcoords.VocabCoordinate]{mango:extcoords.VocabCoordinate}} \newline
      \textbf{multiplicity: 1} \newline 
      TODO : Missing description : please, update your UML model asap.

\pagebreak
\section{Package: extcoords }

  % INSERT FIGURE HERE
  %\begin{figure}[h]
  %\begin{center}
  %  \includegraphics[width=\textwidth]{????.png}
  %  \caption{???}\label{fig:????}
  %\end{center}
  %\end{figure}

  This package contains all \texttt{coords} class extensions

  \subsection{FlagCoord}
  \label{sect:extcoords.FlagCoord}
    Coordinate of a status Measure

    \noindent \textbf{subset} \newline
    \indent   \textbf{role: coords:Coordinate.coordSys} \newline
    \indent   \textbf{type:  mango:extcoords.FlagSys} \newline


    \subsubsection{FlagCoord.status}
      \textbf{vodml-id: extcoords.FlagCoord.status} \newline
      \textbf{type: \hyperref[sect:ivoa]{ivoa:integer}} \newline
      \textbf{multiplicity: 1} \newline 
      Value of the status

  \subsection{FlagState}
  \label{sect:extcoords.FlagState}
    Possible value of a status

    \subsubsection{FlagState.value}
      \textbf{vodml-id: extcoords.FlagState.value} \newline
      \textbf{type: \hyperref[sect:ivoa]{ivoa:integer}} \newline
      \textbf{multiplicity: 1} \newline 
      Status value

    \subsubsection{FlagState.label}
      \textbf{vodml-id: extcoords.FlagState.label} \newline
      \textbf{type: \hyperref[sect:ivoa]{ivoa:string}} \newline
      \textbf{multiplicity: 1} \newline 
      Label attached to that status value

  \subsection{FlagSys}
  \label{sect:extcoords.FlagSys}
    Coordinate system to be used for statur measures.

    \subsubsection{FlagSys.statusLabels}
      \textbf{vodml-id: extcoords.FlagSys.statusLabels} \newline
      \textbf{type: \hyperref[sect:extcoords.FlagState]{mango:extcoords.FlagState}} \newline
      \textbf{multiplicity: 0..*} \newline 
      Composition link to all possible status values for this system

  \subsection{HardnessRatioCoord}
  \label{sect:extcoords.HardnessRatioCoord}
    The hardness ration value with a reference to coordinate frame which is the descrition of the both bands.

    \noindent \textbf{subset} \newline
    \indent   \textbf{role: coords:Coordinate.coordSys} \newline
    \indent   \textbf{type:  mango:extcoords.HardnessRatioSys} \newline


    \subsubsection{HardnessRatioCoord.hardnessRatio}
      \textbf{vodml-id: extcoords.HardnessRatioCoord.hardnessRatio} \newline
      \textbf{type: \hyperref[sect:ivoa]{ivoa:real}} \newline
      \textbf{multiplicity: 1} \newline 
      Hardness ration value

  \subsection{HardnessRatioFrame}
  \label{sect:extcoords.HardnessRatioFrame}
    Hardness ratio frame. Defined by 2 energy bands Eheigh ELow. HR = (Eheigh - Elow)/(Eheigh + Elow) Energy bands are deemed to special photometric filters

    \subsubsection{HardnessRatioFrame.low}
      \textbf{vodml-id: extcoords.HardnessRatioFrame.low} \newline
      \textbf{type: \hyperref[sect:extcoords.PhotFilter]{mango:extcoords.PhotFilter}} \newline
      \textbf{multiplicity: 1} \newline 
      Low energy band

    \subsubsection{HardnessRatioFrame.high}
      \textbf{vodml-id: extcoords.HardnessRatioFrame.high} \newline
      \textbf{type: \hyperref[sect:extcoords.PhotFilter]{mango:extcoords.PhotFilter}} \newline
      \textbf{multiplicity: 1} \newline 
      Heigh energy band

  \subsection{HardnessRatioSys}
  \label{sect:extcoords.HardnessRatioSys}
    Coordinate system for hardness ration. This is basically the definition of the 2 bands used to compute the ratio.

    \noindent \textbf{subset} \newline
    \indent   \textbf{role: coords:PhysicalCoordSys.frame} \newline
    \indent   \textbf{type:  mango:extcoords.HardnessRatioFrame} \newline


  \subsection{LonLatCoordSys}
  \label{sect:extcoords.LonLatCoordSys}
    Space coordinate system

    \noindent \textbf{subset} \newline
    \indent   \textbf{role: coords:PhysicalCoordSys.frame} \newline
    \indent   \textbf{type: coords:SpaceFrame} \newline


    \noindent \textbf{constraint} \newline
    \indent    \textbf{detail: LonLatCoordSys.coordSpace[0] }\newline


  \subsection{LonLatPoint}
  \label{sect:extcoords.LonLatPoint}
    Coordinate of a point on the sky sphere expressed in spherical coordinates.

    \noindent \textbf{subset} \newline
    \indent   \textbf{role: coords:Coordinate.coordSys} \newline
    \indent   \textbf{type:  mango:extcoords.LonLatCoordSys} \newline


    \subsubsection{LonLatPoint.longitude}
      \textbf{vodml-id: extcoords.LonLatPoint.longitude} \newline
      \textbf{type: \hyperref[sect:ivoa]{ivoa:real}} \newline
      \textbf{multiplicity: 1} \newline 
      longitude of the point

    \subsubsection{LonLatPoint.latitude}
      \textbf{vodml-id: extcoords.LonLatPoint.latitude} \newline
      \textbf{type: \hyperref[sect:ivoa]{ivoa:real}} \newline
      \textbf{multiplicity: 1} \newline 
      Latitude of the point

  \subsection{OrbitCoord}
  \label{sect:extcoords.OrbitCoord}
    TODO : Missing description : please, update your UML model asap.

    \noindent \textbf{subset} \newline
    \indent   \textbf{role: coords:Coordinate.coordSys} \newline
    \indent   \textbf{type: coords:SpaceSys} \newline


    \subsubsection{OrbitCoord.period}
      \textbf{vodml-id: extcoords.OrbitCoord.period} \newline
      \textbf{type: \hyperref[sect:ivoa]{ivoa:RealQuantity}} \newline
      \textbf{multiplicity: 1} \newline 
      TODO : Missing description : please, update your UML model asap.

    \subsubsection{OrbitCoord.eccentricity}
      \textbf{vodml-id: extcoords.OrbitCoord.eccentricity} \newline
      \textbf{type: \hyperref[sect:ivoa]{ivoa:real}} \newline
      \textbf{multiplicity: 1} \newline 
      Orbital eccentricity, unitless

  \subsection{PhotFilter}
  \label{sect:extcoords.PhotFilter}
    Photometric filter description, compliant with photDM

    \subsubsection{PhotFilter.name}
      \textbf{vodml-id: extcoords.PhotFilter.name} \newline
      \textbf{type: \hyperref[sect:ivoa]{ivoa:string}} \newline
      \textbf{multiplicity: 1} \newline 
      Filter name

    \subsubsection{PhotFilter.zeroPointFlux}
      \textbf{vodml-id: extcoords.PhotFilter.zeroPointFlux} \newline
      \textbf{type: \hyperref[sect:ivoa]{ivoa:real}} \newline
      \textbf{multiplicity: 1} \newline 
      Zero point flux of the filter

    \subsubsection{PhotFilter.magnitudeSystem}
      \textbf{vodml-id: extcoords.PhotFilter.magnitudeSystem} \newline
      \textbf{type: \hyperref[sect:ivoa]{ivoa:string}} \newline
      \textbf{multiplicity: 1} \newline 
      Magnitude system used by the filter

    \subsubsection{PhotFilter.effectiveWavelength}
      \textbf{vodml-id: extcoords.PhotFilter.effectiveWavelength} \newline
      \textbf{type: \hyperref[sect:ivoa]{ivoa:real}} \newline
      \textbf{multiplicity: 1} \newline 
      Effective wavelength of the filter

    \subsubsection{PhotFilter.unit}
      \textbf{vodml-id: extcoords.PhotFilter.unit} \newline
      \textbf{type: \hyperref[sect:ivoa]{ivoa:Unit}} \newline
      \textbf{multiplicity: 1} \newline 
      Wavelength unit used for that filter

    \subsubsection{PhotFilter.bandWidth}
      \textbf{vodml-id: extcoords.PhotFilter.bandWidth} \newline
      \textbf{type: \hyperref[sect:ivoa]{ivoa:real}} \newline
      \textbf{multiplicity: 1} \newline 
      Band width of the filter

  \subsection{PhotometryCoord}
  \label{sect:extcoords.PhotometryCoord}
    Coordinate value for photometry measures

    \noindent \textbf{subset} \newline
    \indent   \textbf{role: coords:Coordinate.coordSys} \newline
    \indent   \textbf{type:  mango:extcoords.PhotometryCoordSys} \newline


    \subsubsection{PhotometryCoord.luminosity}
      \textbf{vodml-id: extcoords.PhotometryCoord.luminosity} \newline
      \textbf{type: \hyperref[sect:ivoa]{ivoa:real}} \newline
      \textbf{multiplicity: 1} \newline 
      Value of the photometric measure

  \subsection{PhotometryCoordSys}
  \label{sect:extcoords.PhotometryCoordSys}
    TBC with photDM

    \noindent \textbf{subset} \newline
    \indent   \textbf{role: coords:PhysicalCoordSys.frame} \newline
    \indent   \textbf{type:  mango:extcoords.PhotFilter} \newline


  \subsection{RedshiftCoord}
  \label{sect:extcoords.RedshiftCoord}
    TODO : Missing description : please, update your UML model asap.

    \noindent \textbf{subset} \newline
    \indent   \textbf{role: coords:Coordinate.coordSys} \newline
    \indent   \textbf{type:  mango:extcoords.RedshiftSys} \newline


    \subsubsection{RedshiftCoord.redshift}
      \textbf{vodml-id: extcoords.RedshiftCoord.redshift} \newline
      \textbf{type: \hyperref[sect:ivoa]{ivoa:real}} \newline
      \textbf{multiplicity: 1} \newline 
      TODO : Missing description : please, update your UML model asap.

  \subsection{RedshiftSys}
  \label{sect:extcoords.RedshiftSys}
    TODO : Missing description : please, update your UML model asap.

  \subsection{ShapeCoord}
  \label{sect:extcoords.ShapeCoord}
    TODO : Missing description : please, update your UML model asap.

    \noindent \textbf{subset} \newline
    \indent   \textbf{role: coords:Coordinate.coordSys} \newline
    \indent   \textbf{type:  mango:extcoords.ShapeSys} \newline


    \subsubsection{ShapeCoord.shape}
      \textbf{vodml-id: extcoords.ShapeCoord.shape} \newline
      \textbf{type: \hyperref[sect:ivoa]{ivoa:string}} \newline
      \textbf{multiplicity: 1} \newline 
      String serialisation of the shape

  \subsection{ShapeSys}
  \label{sect:extcoords.ShapeSys}
    Coordinate systen to be used for shape measure

    \subsubsection{ShapeSys.frame}
      \textbf{vodml-id: extcoords.ShapeSys.frame} \newline
      \textbf{type: \hyperref[sect:extcoords.ShapeFrame]{mango:extcoords.ShapeFrame}} \newline
      \textbf{multiplicity: 1} \newline 
      Frame of the shape measure. Gives a enumeration of the supported serialisations.

  \subsection{StringCoordinate}
  \label{sect:extcoords.StringCoordinate}
    Coordinate value for \texttt{mango:measures.GenericStringMeasure}

    \subsubsection{StringCoordinate.cval}
      \textbf{vodml-id: extcoords.StringCoordinate.cval} \newline
      \textbf{type: \hyperref[sect:ivoa]{ivoa:string}} \newline
      \textbf{multiplicity: 1} \newline 
      TODO : Missing description : please, update your UML model asap.

  \subsection{VocabCoordinate}
  \label{sect:extcoords.VocabCoordinate}
  Coordinate for \texttt{mango:measures.GenericVocabMeasure}

  \subsection{ShapeFrame}
  \label{sect:extcoords.ShapeFrame}

  Enumeration of the possible options to encode a shape in a string.

  \noindent \underline{Enumeration Literals}
  \vspace{-\parsep}
  \small
  \begin{itemize}
  
    \item[\textbf{MOC}]: \textbf{vodml-id:} extcoords.ShapeFrame.MOC \newline
          \textbf{description:} MOC serialization
    \item[\textbf{STCs}]: \textbf{vodml-id:} extcoords.ShapeFrame.STCs \newline
          \textbf{description:} STCs serialization
  \end{itemize}
  \normalsize
